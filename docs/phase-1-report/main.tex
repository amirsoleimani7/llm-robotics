% !TEX TS-program = xelatex
\documentclass[10pt]{article}

\usepackage[a4paper,margin=2cm]{geometry}
\usepackage{hyperref}
\usepackage{multicol}
\usepackage{xepersian}
\settextfont{B Nazanin}
\setlatintextfont{Latin Modern Roman}

\title{مدل‌های زبانی بزرگ برای رباتیک: یک مرور}
\author{فانلونگ زنگ \quad ونشنگ گن \quad یونگ‌هنگ وانگ \quad نینگ لیو \quad فیلیپ اس. یو}
\date{۱۴۰۳}

\begin{document}
\maketitle

% ======= جعبه‌های بالای صفحه: اطلاعات + چکیده =======
\noindent
\fbox{%
  \parbox[t]{0.32\textwidth}{%
    \textbf{اطلاعات مقاله}\\[0.5ex]
    \textbf{کلیدواژه‌ها:}\\
    مدل‌های زبانی بزرگ\\
    رباتیک\\
    کنترل و تعامل\\
    تصمیم‌گیری\\
    هوش مجسم
  }%
}
\hfill
\fbox{%
  \parbox[t]{0.62\textwidth}{%
    \textbf{چکیده}\\[0.5ex]
    این یک چکیدهٔ نمونه است که به صورت خلاصه هدف، روش و
    نتایج کار را توضیح می‌دهد. می‌توانی این متن را با
    چکیدهٔ واقعی مقاله‌ات جایگزین کنی.
  }%
}

\vspace{1em}
\hrule
\vspace{1em}

% ======= از اینجا به بعد متن دو ستونه =======
\begin{multicols}{2}

\section{مقدمه}
اینجا متن مقدمه قرار می‌گیرد. چیدمان از اینجا به بعد دو ستونه است
و جهت متن توسط \lr{xepersian} راست‌به‌چپ تنظیم شده است.

\begin{latin}
This is an English sentence inside RTL text.
\end{latin}

\section{کارهای مرتبط}
در این بخش کارهای مرتبط توضیح داده می‌شود.

\section{روش پیشنهادی}
اینجا روش پیشنهادی مقاله را می‌توانی بنویسی.

\section{نتیجه‌گیری}
در این بخش جمع‌بندی و نتیجه‌گیری ارائه می‌شود.

\end{multicols}

\end{document}
